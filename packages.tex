\usepackage[nswissgerman]{babel}
\usepackage[utf8]{inputenc}

%============== BibLaTeX ========================
%-------------- Allgemein -----------------------
\usepackage[backend=biber,sorting=none]{biblatex} 
\usepackage{csquotes}
%------------- Neudefinitionen ------------------
%\DefineBibliographyStrings{german}{%
%urlseen = {aufgerufen am},
%}

% Multi columns
\usepackage{multicol}
\usepackage{tikz}
\usetikzlibrary{decorations.pathmorphing}
\usetikzlibrary{arrows,matrix}

% Colors
\usepackage{colortbl}
\usepackage{xcolor}

\definecolor{magenta}{RGB}{90,56,165}
\definecolor{dockerBlue}{RGB}{90,177,241}
\definecolor{tblgray}{rgb}{0.9,0.9,0.9}
\colorlet{tblyellow}{yellow!30}
\colorlet{tblgreen}{green!20}
\colorlet{tblred}{red!20}
\colorlet{tblmagenta}{magenta!20}
\colorlet{tblblue}{blue!20}


% Mathematics
\usepackage{amsmath,amssymb}
\usepackage{amsmath}
\usepackage{esint}
\usepackage{amsfonts}
\usepackage{mathtools}

% floating tables and images better
\usepackage{float}

% Lists 
\usepackage{enumitem}
\setlist{nosep}

% Dashed-Tables
\usepackage{arydshln}
\setlength\dashlinedash{0.2pt}
\setlength\dashlinegap{1.5pt}
\setlength\arrayrulewidth{0.3pt}

%============== Code-Highlighting ===============
\usepackage{listings}

%Enter your programming language in the subject here
\lstset{frame=tb,
  aboveskip=3mm,
  belowskip=3mm,
  showstringspaces=false,
  columns=flexible,
  basicstyle={\small\ttfamily},
  numbers=none,
  numberstyle=\tiny\color{gray},
  keywordstyle=\color{blue},
  commentstyle=\color{dkgreen},
  stringstyle=\color{mauve},
  breaklines=true,
  breakatwhitespace=true,
  tabsize=3
}

% Hyperlinkgs (should be at the end, but before geometry)
\usepackage[pdftex,
pdfauthor={TBD},
pdftitle={TBD},
hidelinks]{hyperref}

% Geometry (should be at the end)
\usepackage[landscape]{geometry}
\makeatletter

%\colorbox[HTML]{e4e4e4}{\makebox[\textwidth-2\fboxsep][l]{texto}


%You can also use this:
\newcommand{\boxnode}[2]
{
    {
    %------------ Styling ----------------
    
    %------------ Content ----------------
    \begin{tikzpicture}
    \node [mybox] (box){%
        \begin{minipage}{0.3\textwidth}
                #2
        \end{minipage}
    };
    %------------  Header ---------------------
    \node[fancytitle, right=10pt] at (box.north west) {#1};
    \end{tikzpicture}
    }
}


\newcommand{\yellowBorderNode}[2]
{
    {
    %------------ Styling ----------------
    \tikzstyle{mybox} = [draw=yellow, fill=white, very thick, rectangle, rounded corners, inner sep=10pt, inner ysep=10pt]

    %------------ Content ----------------
    \begin{tikzpicture}
    \node [mybox] (box){%
        \begin{minipage}{0.3\textwidth}
                #2
        \end{minipage}
    };
    %------------  Header ---------------------
    \node[fancytitle, right=10pt] at (box.north west) {#1};
    \end{tikzpicture}
    }
}

\newcommand{\greenBorderNode}[2]
{
    {
    %------------ Styling ----------------
    \tikzstyle{mybox} = [draw=green, fill=white, very thick, rectangle, rounded corners, inner sep=10pt, inner ysep=10pt]

    %------------ Content ----------------
    \begin{tikzpicture}
    \node [mybox] (box){%
        \begin{minipage}{0.3\textwidth}
                #2
        \end{minipage}
    };
    %------------  Header ---------------------
    \node[fancytitle, right=10pt] at (box.north west) {#1};
    \end{tikzpicture}
    }
}


\newcommand{\redBorderNode}[2]
{
    {
    %------------ Styling ----------------
    \tikzstyle{mybox} = [draw=red, fill=white, very thick, rectangle, rounded corners, inner sep=10pt, inner ysep=10pt]

    %------------ Content ----------------
    \begin{tikzpicture}
    \node [mybox] (box){%
        \begin{minipage}{0.3\textwidth}
                #2
        \end{minipage}
    };
    %------------  Header ---------------------
    \node[fancytitle, right=10pt] at (box.north west) {#1};
    \end{tikzpicture}
    }
}

\newcommand{\magentaBorderNode}[2]
{
    {
    %------------ Styling ----------------
    \tikzstyle{mybox} = [draw=magenta, fill=white, very thick, rectangle, rounded corners, inner sep=10pt, inner ysep=10pt]

    %------------ Content ----------------
    \begin{tikzpicture}
    \node [mybox] (box){%
        \begin{minipage}{0.3\textwidth}
                #2
        \end{minipage}
    };
    %------------  Header ---------------------
    \node[fancytitle, right=10pt] at (box.north west) {#1};
    \end{tikzpicture}
    }
}

\newcommand{\blueBorderNode}[2]
{
    {
    %------------ Styling ----------------
    \tikzstyle{mybox} = [draw=blue, fill=white, very thick, rectangle, rounded corners, inner sep=10pt, inner ysep=10pt]

    %------------ Content ----------------
    \begin{tikzpicture}
    \node [mybox] (box){%
        \begin{minipage}{0.3\textwidth}
                #2
        \end{minipage}
    };
    %------------  Header ---------------------
    \node[fancytitle, right=10pt] at (box.north west) {#1};
    \end{tikzpicture}
    }
}
